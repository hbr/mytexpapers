\section{Typing}

\subsection{Typing Rules}


\begin{theorem}
  \label{generalizedconversion}
  Generalized conversion
  $$
  \ruleh
  {
    \Gamma \vdash t : U
    \\
    T \reducestar U
  }
  {
    \Gamma \vdash t : T
  }
  $$

  \begin{proof}
    By induction on $T \reducestar U$
  \end{proof}
\end{theorem}



Free variables lemma

Transitivity lemma

Arbitrary type expansion


\subsection{Substitution Lemma}


Thinning lemma

\subsection{Generation Lemmata}


\subsection{Subject Reduction Theorem}

\begin{theorem}
  \label{subjectreduction}
  Reduction preserves typing i.e.
  $$
  \ruleh
  {\Gamma \vdash t : T
    \\
    t \reduce u
  }
  {\Gamma \vdash u : T}
  $$
  \begin{proof}
    We define $\Gamma \reduce \Delta$ if $\Delta$ is $\Gamma$ where one type
    has been reduced. Then we proof the stronger theorem
    $$
    \ruleh
    {\Gamma \vdash t : T}
    {
      \rulev
      {t \reduce u}
      {\Gamma \vdash u : T}
      \land
      \rulev
      {\Gamma \reduce \Delta}
      {\Delta \vdash t : T}
    }
    $$
    by induction on $\Gamma \vdash t : T$

    \begin{description}

    \item[Axiom] Not possible, because the context is empty and terms are
      sorts and neither of them can be reduced.

    \item[Variable]
      $$
      \ruleh
      {\Gamma \vdash A:s \quad x \notin \Gamma}
      {\Gamma,x:A \vdash x:A}
      $$

      The left part is vacuously satisfied because a variable is in normal
      form and cannot be reduced.

      For the right part of the goal assume $\Gamma,x:A$ reduces to another
      context. There are two cases possible
      \begin{enumerate}

      \item $\Gamma,x:A \reduce \Delta,x:A$ where $\Gamma \reduce
        \Delta$. From the induction hypothesis we can conclude $\Delta \vdash
        A:s$ and therefore $\Delta,x:A \vdash x:A$ is valid.

      \item  $\Gamma,x:A \reduce \Gamma,x:B$ where $A \reduce B$. From the
        induction hypothesis we get $\Gamma \vdash B: s$ and therefore
        $\Gamma,x:B \vdash x:B$ is valid.
      \end{enumerate}

    \item[Weaken]
      $$
      \ruleh
      {\Gamma \vdash A:s \quad x \notin \Gamma \quad \Gamma \vdash t : T}
      {\Gamma,x:A \vdash t : T}
      $$

      The left part of the goal is an immediate consequence of the induction
      hypothesis derived from $\Gamma \vdash t : T$. The right part needs the
      same case split as for the previous rule. For both cases the right part
      of the goal can be derived from the induction hypothesis.


    \item[Product]
      $$
      \rulev
      { \Gamma \vdash A:s_1
        \\
        \Gamma,x:A \vdash B: s_2
        \\
        r(s_1,s_2,s_3)}
      {\Gamma \vdash \Pi x^A. B : s_3}
      $$

      For the left part of the goal assume that $\Pi x^A. B$ reduces. The
      either  $A$ or $B$ reduces. From the induction hypotheses we conclude
      that the reduced terms have the same sort. So the reduce product has the
      same sort.

      For the right part of the goal assume that the context $\Gamma$
      reduces. The induction hypotheses immediately give that the terms $A$
      and $B$ have the same sorts, therefore the product has the same sort
      under the reduced context.

    \item[Abstraction]
      $$
      \rulev
      { \Gamma \vdash \Pi x^A. B : s
        \\
        \Gamma,x:A \vdash e: B
      }
      {\Gamma \vdash \lambda x^A. e : \Pi x^A. B}
      $$

      Left part: Assume that $\lambda x^A. e$ reduces. I.e. either $A$ or $e$
      reduce. From the induction hypotheses we conclude the same types.

      Right part: Assume that $\Gamma$ reduces and use the induction
      hypotheses to prove the goal.


    \item[Application]
      $$
      \rulev
      { \Gamma \vdash f : \Pi x^A. B
        \\
        \Gamma \vdash a: A
      }
      {\Gamma \vdash f a : B[x:=a]}
      $$

      Right part: Immediate consequence of the induction hypotheses.

      Left part: Assume that $f a$ reduces and prove that the reduct has type
      $B[x:=a]$. There are 3 cases to consider. Either $f$ reduces or $a$
      reduces or $f$ is a lambda term and $f a$ is a redex.
      \begin{enumerate}

      \item $f \reduce g$: We get $g: \Pi x^A. B$ by the induction hypothesis
        and $g a : B[x:=a]$.

      \item $a \reduce b$: Same reasoning.

      \item $(\lambda x^{A'} .e) a \reduce e[x:=a]$ where
        $\lambda x^{A'} .e: \Pi x^A. B$:
        Goal: $$\Gamma \vdash e[x:=a] : B[x:=a].$$
        By the subsitution lemma we
        generate the subgoal $$\Gamma,x:A \vdash e : B.$$

        From the generation lemma for abstractions we get
        $$
        \exists s' B'.
        \left[
          \begin{array}{l}
          \Gamma \vdash\Pi x^{A'}.B' : s' \quad\land
          \\
          \Gamma,x:A' \vdash e:B' \quad\land
          \\
          \Pi x^A. B \reducestar \Pi x^{A'}.B'
          \end{array}
          \right]
        $$
        Since reduction preserves product we get $A \reducestar A'$ and
        $b \reducestar B'$. Together with $\Gamma,x:A' \vdash e:B'$ and the
        generalized conversion rule~\ref{generalizedconversion} we prove the
        subgoal.
      \end{enumerate}


    \item[Conversion]
      $$
      \rulev
      { \Gamma \vdash t : U
        \\
        \Gamma \vdash T : s
        \\
        T \reduce U \lor U \reduce T
      }
      {\Gamma \vdash t: T}
      $$
      Left and right part of the goal are immediate consequences of the
      induction hypothesis and the conversion rule.
    \end{description}
  \end{proof}
\end{theorem}





MISSING: Condensing lemma!!
