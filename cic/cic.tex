\documentclass[12pt]{article}
\usepackage{amsmath,amsthm,amssymb,xcolor,graphicx,tikz,listings}
\usepackage[pdftex]{hyperref}
\usetikzlibrary{positioning}


\def\imp{\Rightarrow}
\def\ind{\text{Ind}}
\def\abold{\mathbf{a}}
\def\Abold{\mathbf{A}}
\def\bbold{\mathbf{b}}
\def\Bbold{\mathbf{B}}
\def\fbold{\mathbf{f}}
\def\Fbold{\mathbf{F}}
\def\Ibold{\mathbf{I}}
\def\pbold{\mathbf{p}}
\def\Pbold{\mathbf{P}}
\def\tbold{\mathbf{t}}
\def\Tbold{\mathbf{T}}
\def\ubold{\mathbf{u}}
\def\Ubold{\mathbf{U}}
\def\xbold{\mathbf{x}}
\def\ybold{\mathbf{y}}
\def\zbold{\mathbf{z}}

\def\Gammabold{\mathbf{\Gamma}}
\def\Deltabold{\mathbf{\Delta}}

\def\Prop{\mathbb{P}}
\def\Set{\mathbb{S}}
\def\Type{\mathbb{T}}

\def\Ar{\text{Ar}}
\def\CT{\text{CT}}
\def\case{\text{case}}
\def\fix {\text{fix}}
\def\FV{\text{FV}}
\def\freevars{\text{FV}}
\def\strictpos{\text{SP}}
\def\Ind{\text{Ind}}
\def\ind{\text{ind}}
\def\Pos{\text{Pos}}

\newcommand{\Induct}[2]{\mathbb{I}^#1_#2}
\newcommand{\Con}[3]{\mathbb{C}^{#1 #2}_#3}
\newcommand{\ruleh}[2]{\begin{array}{c} #1 \\ \hline #2\end{array}}
\newcommand{\rulev}[2]{\begin{array}{l} #1 \\ \hline #2\end{array}}

\theoremstyle{definition} \newtheorem{definition}{Definition}[section]
\theoremstyle{definition} \newtheorem{theorem}[definition]{Theorem}
\theoremstyle{definition} \newtheorem{lemma}[definition]{Lemma}


\begin{document}


\title{Calculus of Inductive Constructions}
\author{Helmut Brandl \\ \scriptsize (firstname dot lastname at gmx dot net)}
\date{}


\maketitle

\abstract{
}

\tableofcontents







\section{Basic Definitions}
\label{sec:terms}




\paragraph{Sorts} Sorts are defined by the grammar
$$ s  ::=  \Prop \mid \Type_i.$$
We use $\Set$ is an abbreviation for $\Type_0$.




\paragraph{Terms} Let $s$ range over sorts, $x$ range over variables and $i$
$j$ range over numbers. The terms are defined by the grammar
$$
\begin{array}{llll}
  t & ::= & s                    & \text{sort} \\
    &\mid & x                    &\text{variable}\\
    &\mid & t t                  &\text{application}\\
    &\mid & \lambda x^t. t       &\text{abstraction} \\
    &\mid & \Pi x^t. t           &\text{product} \\
    &\mid & \Induct{i}{\ind}     &\text{$i$th inductive type of the family
                                   $\ind$}\\
    &\mid & \Con{j}{i}{\ind}     &\text{$j$th constructor of $i$th type of
                                   the family $\ind$}\\
    &\mid & \case(t,t,\tbold)    &\text{pattern match}\\
    &\mid & \fix (i,\Fbold)      & \text{fixpoint}
\end{array}
$$
where
\begin{itemize}
\item
  $\ind = (\Delta_P,\Delta_I,\Deltabold_C)$ a family of inductive types where
  $\Delta_P$ represents the parameters of the family, $\Delta_I$ represents
  the array of inductive types of the family and $\Deltabold_C$ represents for
  each inductive type an array of constructor types,

\item
  $\Delta = [x_1:T_1, \,\cdots]$ is an array of variable declarations,

\item $\Deltabold = [\Delta_1, \,\cdots]$ is an array of arrays of variable
  declarations.

\item $\Fbold = [x_1:T_1 := t_1/k_1, \,\cdots ]$ is an array of quadruples with
  the variable $x_j$ and its type $T_j$, a definition term $t_j$ and the
  decreasing argument number $k_j$, where the terms $t_j$ might contain
  any variable $x_l$.
\end{itemize}





\paragraph{Reduction}We have beta reduction, pattern match and fixpoint
reduction.
$$
\begin{array}{lll}
  (\lambda x^A.e) a &\triangleright  &e[x:=a] \\

  \case(\Con{j}{i}{\ind}\pbold\abold,r,\fbold)
                    &\triangleright   & f_j \abold \\
  \fix(i,\Fbold) \abold (\Con{j}{k}{\ind} \bbold)
                    &\triangleright
               &t_i[f_j:=\fix(j,\Fbold)] \abold (\Con{j}{k}{\ind} \bbold)
\end{array}
$$







\paragraph{Inductive Types}


\begin{definition}
  A positive term over the variable $P$ has the form
  $\Pi \xbold^\Abold. P \bbold$ where $P$ does not occur in $\Abold$. The
  product part might be empty.
\end{definition}



\begin{definition}
  An arity of sort $s$ has the form $\Pi \xbold^\Abold. s$ where the product
  part might be empty.
\end{definition}


\begin{definition}
  A constructor type for the type variable $I$ representing an inductive type
  has the form $\Pi \xbold^\Abold. I \abold$ where the product part might be
  empty. The types $\Abold$ are called the argument types of the constructor
  type.
\end{definition}


\begin{definition}
  An inductive family $\ind = (\Delta_P,\Delta_I,\Deltabold_C)$ is valid if
  the following conditions are satisfied:
  \begin{enumerate}

  \item In the inductve types $\Delta_I = [I_0:T_0,I_1:T_1,\cdots]$ all $T_i$
    must convert to an arity of some sort. The types $\Tbold$ are the
    inductive types of the family.

  \item $|\Delta_I| = |\Deltabold_C|$ i.e. there is one constructor array for
    each inductive type of the family.

  \item In the constructors $\Delta_{Ci} = [c_0:T_0,c_1:T_1,\cdots]$ for the
    inductive type variable $I_i$ of the family, all types $T_j$ of the
    constructors convert to the form $\Pi \xbold^\Abold. I_i\bbold$ i.e. a
    constructor type for the variable $I_i$. In all constructor argument types
    $\Abold$ all inductive type variables $\Ibold$ either do not occur or
    occur only positively.


  \item A constructor argument whose type contains an inductive type variable
    of the family are called recursive arguments.

  \item A parameter of the family is called a positive parameter if it occurs
    in all constructor argument types of all constructors of the family only
    positively.

  \item A constructor argument whose type contains a positive parameter is a
    potentially recursive argument.
  \end{enumerate}
\end{definition}






\section{Draft}

\begin{definition}
  Subtype relation:
  \begin{enumerate}

  \item $\Prop < \Type_0$

  \item $\Type_i < \Type_{i'}$

  \item $\ruleh{T < U}{\forall x^A.T < \forall x^A.U}$
  \end{enumerate}
\end{definition}
Note: The subtype relation only applies to sorts and arities. Arbitrary sort
terms are not included. Reduction shall be reflected in the typing relation.

\begin{definition}
  Typing relation:

  \begin{itemize}
  \item Basics:
    \begin{description}

    \item[Axiom] $[]\vdash \Prop:\Type_0 \quad\quad
      []\vdash \Type_i: \Type_{i'}$

    \item[Variable]
      $\ruleh{\Gamma\vdash T: s}{\Gamma,x:T\vdash x:T}$

    \item[Weaken]
      $\ruleh{
        \Gamma\vdash A:B \\
        \Gamma\vdash T: s}
      {\Gamma,x:T\vdash A:B}$

    \end{description}
  \item Product formation:
    \begin{description}
    \item[Product-$\Prop$]
      $\ruleh
      {\Gamma\vdash A:s \quad \Gamma,x:A\vdash B:\Prop}
      {\Gamma\vdash \Pi x^A.B : \Prop}$

    \item[Product-$s$]
      $\ruleh
      {\Gamma\vdash A:s \quad \Gamma,x:A\vdash B:s}
      {\Gamma\vdash \Pi x^A.B : s}$

    \end{description}

  \item Functions:
    \begin{description}
    \item[Application]
      $\ruleh
      {\Gamma\vdash f: \Pi x^A.B \quad \Gamma\vdash a : A}
      {\Gamma\vdash f a: B[x:=a]}$

    \item[Abstraction]
      $\ruleh
      {\Gamma\vdash \Pi x^A.B : s \quad \Gamma,x:A\vdash e: B}
      {\Gamma\vdash \lambda x^A.e : \Pi x^A.B}$
    \end{description}

  \item Inductive families
    \begin{description}
    \item[Ind-Type]
      $$\rulev
      {\ind = (\Delta_P, \Delta_I, \Deltabold_C) \text{ is valid}
        \\
        \forall i. \Gamma,\Delta_P\vdash I_i : T_i
        \\
        \forall ij. \Gamma,\Delta_P,\Delta_I \vdash c_{ji}:C_{ji}
      }
      {\Gamma\vdash \Induct{i}{\ind}
        : \Pi \ybold^\Pbold . T_i}$$

    \item[Constructor]
      $$\rulev
      {\forall k. \Gamma\vdash \Induct{k}{\ind} : \Pi \ybold^\Pbold . T_k
      }
      {\Gamma\vdash \Con{j}{i}{\ind}:
        \Pi\ybold^\Pbold. C_{ji}[I_k:= \Induct{k}{\ind}]}$$
    \end{description}


  \item Reduction, expansion and subtyping:
    \begin{description}
    \item [Reduction]
      $\ruleh{\Gamma \vdash t:T \quad T \to U}{\Gamma\vdash t:U}$

    \item [Expansion]
      $\ruleh
      {\Gamma \vdash t:U \quad T \to U \quad\Gamma\vdash T:s}
      {\Gamma\vdash t:T}$

    \item [Subtype]
      $\ruleh
      {\Gamma \vdash t:T \quad T < U}
      {\Gamma\vdash t:U}$
    \end{description}
  \end{itemize}
\end{definition}

The product formation rule of PTS says
$\frac{r(s_1,s_2,s_3;\, \Gamma\vdash A:s_1;\: \Gamma,x:A\vdash
  B:s_2}{\Gamma\vdash (\Pi x:A.B): s_3}$. The calculus of constructions has
the rule $r(*,*,*)$. This allows me using the first and the second line to
derive the third. However I had a typo in my post. The type in the third line
had been missing which I added now.

\begin{definition}
  Classification of terms:
  \begin{enumerate}

  \item An arity is a term of the form $\Pi \xbold^\Abold.s$ where $s$ is
    a sort. The product part might be empty. I.e. all sorts are arities.

  \item A sort term is a term which reduces to an arity.

  \item A type term is a term whose type is a sort term.

  \item An object term is a term whose type is a type term.
  \end{enumerate}
\end{definition}

With this definition all inductive types are type terms because their types
reduce to an arity i.e. they are type terms. This implies that all terms of
the form $\Induct{i}{\ind}$ are type terms. And all terms of the form
$\Con{j}{i}{\ind}$ object terms because their types are type terms.



\section{Scratch}


$[\![x+1]\!], (\!|x|\!)$.

{\scriptsize\texttt{1234567890 Inductive}}

{\scriptsize\tt 123456789 123456789 123456789 123456789 123456789 123456789 123456789}

\lstset{
  morekeywords={Inductive,all,class,create,end},
  keywordstyle=\color{blue}
}

\lstdefinelanguage{coq}
{ basicstyle=\footnotesize\tt,
  columns=flexible,
  keywords={Inductive,Definition,Proof},
  keywordstyle=\color{blue}
}

\lstdefinelanguage{alba}
{ basewidth=0.45em,
  basicstyle=\scriptsize\tt,
  %columns=flexible,
  keywords={all,class,create,end},
  keywordstyle=\color{blue}
}

\lstnewenvironment{alba} {\lstset{language=alba}} {}

\begin{alba}
  class
      List(A)
  create
      []
      (^) (A,List(A))
  end

  class
      (*) (A:Any, r:Endorelation(A)): Endorelation(A)
  create
      reflexive(a:A): (*r)(a,a)
      next(a,b,c:A):  (*r)(a,b) -> r(b,c) -> (*r)(a,c)
  end
\end{alba}

{\footnotesize
  \begin{lstlisting}[
    basewidth=0.45em,
    basicstyle=\scriptsize\ttfamily,
    columns=fixed]
 123456789 123456789 123456789 123456789 123456789 123456789 123456789
    Inductive tree (A:Type): Type :=
    | Node: A -> list (tree A) -> tree A.

    Inductive list (A:Type): Type :=
    | Nil:  list A
    | Cons: A -> list A -> list A.

    Inductive star (A:Type) (r:Endorelation A): Endorelation A :=
    | start: forall a, star A r a a
    | next:  forall a b c, star A r a b -> r b c -> star A r a c

    class
        (*) (A:Any): Endorelation(A) -> Endorelation(A)
    create
        start(a:A): (*r)(a,a)
        next(a,b,c:A): (*r)(a,b) -> r(b,c) -> (*r)(a,c)
    end
    -- without names
    class
        (*) (A:Any): Endorelation(A) -> Endorelation(A)
    create
        all(a:A) (*r)(a,a)
        all(a,b,c:A) (*r)(a,b) -> r(b,c) -> (*r)(a,c)
    end
\end{lstlisting}
}
\end{document}
