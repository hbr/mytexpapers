\documentclass[12pt]{article}
\usepackage{amsmath,amsthm,amssymb,xcolor,graphicx,tikz,listings}
\usepackage[pdftex]{hyperref}
\usetikzlibrary{positioning}

\def\Kinds{\text{Kinds}}
\def\SN{\text{SN}}
\def\SNA{\text{SNA}}
\def\SNS{\text{SNS}}
\def\Types{\text{Types}}

\def\imp{\Rightarrow}
\def\reduce{\,\triangleright\,}

\def\abold{\mathbf{a}}
\def\Abold{\mathbf{A}}
\def\bbold{\mathbf{b}}
\def\Bbold{\mathbf{B}}
\def\fbold{\mathbf{f}}
\def\Fbold{\mathbf{F}}
\def\pbold{\mathbf{p}}
\def\Pbold{\mathbf{P}}
\def\tbold{\mathbf{t}}
\def\Tbold{\mathbf{T}}
\def\ubold{\mathbf{u}}
\def\Ubold{\mathbf{U}}

\def\Prop{\mathbb{P}}
\def\Set{\mathbb{S}}
\def\Type{\mathbb{T}}

\def\freevars{\text{FV}}

\newcommand{\goalbg}[1]{\colorbox{lightgray}{$#1$}}

\newcommand{\ruleh}[2]{\begin{array}{c} #1 \\ \hline #2\end{array}}
\newcommand{\rulev}[2]{\begin{array}{l} #1 \\ \hline #2\end{array}}

\theoremstyle{definition} \newtheorem{definition}{Definition}[section]
\theoremstyle{definition} \newtheorem{theorem}[definition]{Theorem}
\theoremstyle{definition} \newtheorem{lemma}[definition]{Lemma}



\renewcommand{\descriptionlabel}[1]{\hspace{5mm}\emph{#1}}


\begin{document}


\title{Typed Lambda Calculus}
\author{Helmut Brandl \\ \scriptsize (firstname dot lastname at gmx dot net)}
\date{}


\maketitle

\abstract{
}

\tableofcontents




\newpage
\section{Pure Type Systems}

Convention: We use $s,s_1,s_2,\ldots$ to range over sorts,
$x,y,z,\ldots,x_1,x_2,\ldots$ to range over variables, $\alpha,\beta,\ldots$
to range over type variables and arbitrary letters $a,b,c,t,u,v,A,B,C,\ldots$
to range over arbitrary terms. Capital letters are used preferable to range
over terms which are meant to represent types.

\begin{definition}
  Typed lambda terms are defined by the grammar
  $$
  \begin{array}{llll}
    t &::=& s                 &\text{sort}
    \\
      &\mid& x                &\text{variable}
    \\
      &\mid& t t              &\text{application}
    \\
      &\mid& \lambda x^t.t    &\text{abstraction}
    \\
      &\mid& \Pi x^t.t        &\text{product}
  \end{array}
  $$
  where $s$ ranges over sorts, $x$ ranges over variables and $t$ ranges over
  terms.
\end{definition}


\begin{definition}
  The reduction relation $\reduce$ is defined as the compatible closure
  of
  $$
  (\lambda x^A. e) a \reduce e[x:=a]
  $$
\end{definition}





\begin{definition}
  The set of kinds is defined inductively by the rules
  \begin{enumerate}

  \item $s \in \Kinds$

  \item $\ruleh{k \in \Kinds}{\Pi x^A.k \in \Kinds}$

  \item $\ruleh
    {k \in \Kinds \quad l \reduce k}
    {l \in \Kinds}$
  \end{enumerate}

\end{definition}



\begin{definition}
  The set of Types is defined inductively by the rules
  \begin{enumerate}

  \item $\alpha \in \Types$

  \item $\ruleh{B \in \Types}{\Pi x^A.B \in \Types}$

  \item $\ruleh
    {U \in \Types \quad T \reduce U}
    {T \in \Types}$
  \end{enumerate}

\end{definition}

Note that the set $\Kinds$ and the set $\Types$ are just syntactical
categories. There is no guarantee that the terms in these sets are wellformed
according to any typing relation. The terms just satisfy some syntax rules.


\begin{definition}
  A context $\Gamma = [x_1:T_1, \ldots, x_n:T_n]$ is a list of variables
  together with their types. The types $T_i$ are lambda terms. No duplicate
  variables are allowed. We use the notation
  $\Gamma,x:T = [x_1:T_1, \ldots, x_n:T_n, x:T]$ to introduce the new fresh
  variable $x$ with its type $T$ into the context $\Gamma$.
\end{definition}





\begin{definition}
  A pure type system has a set of sorts $S$, a set of axioms
  $\{(s_1,s_2), \ldots\}$ and a set of rules $\{(s_1,s_2,s_3), \ldots \}$ and
  the inductively defined typing relation $\Gamma \vdash t: T$ where $\Gamma$
  is a context and $t$ and $T$ are terms.
  \begin{description}

  \item[Axiom] $$\ruleh{ \text{axiom}(s_1,s_2)}{\Gamma \vdash s_1:s_2}$$

  \item[Variable]
    $$\ruleh{\Gamma \vdash A:s\quad x \notin \Gamma}{\Gamma,x:A \vdash x:A}$$


  \item[Weaken]
    $$\ruleh
    {\Gamma \vdash A:s \quad  \Gamma \vdash t:T   \quad x \notin \Gamma}
    {\Gamma,x:A \vdash t:T}$$


  \item[Product]
    $$\ruleh
    {\Gamma \vdash A:s_1 \quad
      \Gamma,x:A \vdash B:s_2 \quad
      \text{rule}(s_1,s_2,s_3)}
    {\Gamma \vdash \Pi x^A . B : s_3}
    $$

  \item[Abstraction]
    $$\ruleh
    {\Gamma \vdash \Pi x^A.B: s \quad
      \Gamma,x:A \vdash e:B
    }
    {\Gamma \vdash \lambda x^A.e : \Pi x^A . B}
    $$

  \item[Application]
    $$\ruleh
    {\Gamma \vdash f: \Pi x^A.B \quad
      \Gamma \vdash a:A
    }
    {\Gamma \vdash f a : B[x:=a]}
    $$

  \item[Type Reduction]
    $$\ruleh
    {\Gamma \vdash t: T \quad
      \Gamma\vdash U: s \quad
      (T \reduce U) \lor (U \reduce T)
    }
    {\Gamma \vdash t: U}
    $$
  \end{description}
\end{definition}






\begin{theorem} Substitution Lemma
  $$
  \rulev
  {\Gamma\vdash a:A
    \\
    \Gamma,x:A,\Delta\vdash t:T
  }
  {\Gamma,\Delta[x:=a] \vdash t[x:=a] : T[x:=a]}
  $$
  \begin{proof} Assume $\Gamma\vdash a:A$ and proof the goal by induction on
    $\Gamma,x:A,\Delta\vdash t:T$. We use the abbreviation $V' := V[x:=a]$.
    \begin{description}
    \item[Product]
      $$
      \begin{array}{l|l}
        \Gamma,x:A,\Delta\vdash B:s_1
        & \Gamma,\Delta' \vdash B': s_1
        \\
        \Gamma,x:A,\Delta,y:B \vdash C:s_2
        & \Gamma,\Delta',y:B' \vdash C': s_2
        \\
        \hline
        \Gamma,x:A,\Delta \vdash \Pi y^B.C: s_3
        & \Gamma,\Delta' \vdash (\Pi y^B.C)': s_3
      \end{array}
      $$
    \end{description}
    TBD!!!!!?????
  \end{proof}
\end{theorem}


\begin{theorem} The type $T$ of a term $t$ is either a sort or its type is a
  sort
  $$
  \ruleh
  {\Gamma \vdash t:T}
  {\exists s. T = s \lor \Gamma\vdash T:s}.
  $$

  \begin{proof}
    By induction on $\Gamma\vdash t:T$.
    \begin{description}

    \item[Axiom] Trivial.


    \item[Variable]
      $\ruleh
      {\Gamma\vdash B:s}
      {\Gamma,x:B \vdash x:B}
      $. Trivial, witness $s$.

    \item[Weaken]
      $$
      \begin{array}{l|l}
        \Gamma \vdash A:s &
        \\
        \Gamma \vdash t:T & \exists s. T = s \lor \Gamma \vdash T:s
        \\
        \hline
        \Gamma,x:A \vdash t:T   & \exists s.
                                  T = s \lor \Gamma,x:A \vdash T:s
      \end{array}
      $$
      The goal is a consequence of the induction hypothesis and an
      application of the weakening rule in case $\Gamma\vdash T:s$.

    \item[Product] Trivial, because the type of a product is always a sort.

    \item[Abstraction] Trivial, because the type of an abstraction is a
      product whose type is always a sort.

    \item[Application]
      $$
      \begin{array}{l|l}
        \Gamma\vdash f:\Pi x^A.B & \exists s. \Pi x^A.B = s \lor
                                   \Gamma\vdash \Pi x^A.B: s
        \\
        \Gamma\vdash a:A
        \\
        \hline
        \Gamma\vdash f a:B[x:=a] & \exists s. B[x:=a] = s \lor
                                   \Gamma\vdash B[x:=s]: s
      \end{array}
      $$
      The type of a product is always a sort. Therefore
      $\Gamma\vdash \Pi x^A.B : s $ must be valid for some sort $s$. The
      product can only be formed if $\Gamma,x:A \vdash B:s$ is valid for some
      sort $s$. \emph{MISSING THEOREM!!}. Since substitution does not change
      the type we get $B[x:=a]: s$.
    \item[Reduction]
      $$
      \begin{array}{l|l}
        \Gamma\vdash t:T \\
        \Gamma\vdash U: s \\
        (T \reduce U) \lor (U \reduce T) \\
        \hline
        \Gamma\vdash t:U & \exists s. U = s \lor \Gamma\vdash U:s
      \end{array}
      $$
      Trivial by second premise.
    \end{description}
  \end{proof}
\end{theorem}


\end{document}
