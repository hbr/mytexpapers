\documentclass[12pt]{article}
\usepackage{amsmath,amsthm,amssymb,xcolor,graphicx,tikz,listings}
\usepackage[pdftex]{hyperref}
\usetikzlibrary{positioning}


\def\imp{\Rightarrow}
\def\ind{\text{Ind}}
\def\abold{\mathbf{a}}
\def\Abold{\mathbf{A}}
\def\bbold{\mathbf{b}}
\def\Bbold{\mathbf{B}}
\def\fbold{\mathbf{f}}
\def\Fbold{\mathbf{F}}
\def\pbold{\mathbf{p}}
\def\tbold{\mathbf{t}}
\def\Tbold{\mathbf{T}}
\def\ubold{\mathbf{u}}
\def\Ubold{\mathbf{U}}

\def\Prop{\mathbb{P}}
\def\Set{\mathbb{S}}
\def\Type{\mathbb{T}}

\def\freevars{\text{FV}}
\def\strictpos{\text{SP}}
\def\case{\text{case}}
\def\fix {\text{fix}}

\newcommand{\ruleh}[2]{\begin{array}{c} #1 \\ \hline #2\end{array}}
\newcommand{\rulev}[2]{\begin{array}{l} #1 \\ \hline #2\end{array}}

\theoremstyle{definition} \newtheorem{definition}{Definition}[section]
\theoremstyle{definition} \newtheorem{theorem}[definition]{Theorem}
\theoremstyle{definition} \newtheorem{lemma}[definition]{Lemma}


\begin{document}


\title{Calculus of Inductive Constructions}
\author{Helmut Brandl \\ \scriptsize (firstname dot lastname at gmx dot net)}
\date{}


\maketitle

\abstract{
}

\tableofcontents




\section{Terms}
\label{sec:terms}


\paragraph{Sorts} Sorts are defined by the grammar
$$ s  ::=  \Prop \mid \Set \mid \Type_i$$
where $\Set$ is an abbreviation for $\Type_0$.

\paragraph{Terms} Let $s$ range over sorts and $x$ range over variables then
terms are defined by the grammar
$$
\begin{array}{llll}
  t & ::= & s & \text{sort} \\
    &\mid & x & \text{variable}\\
    &\mid & t t &   \text{application}\\
    &\mid & \lambda (x:t) t &\text{abstraction} \\
    &\mid & \forall (x:t) t &\text{product} \\
    &\mid & \case(t,t,\tbold) &\text{pattern match}\\
    &\mid & \fix (i,\Fbold) & \text{fixpoint}
\end{array}
$$
where $\Fbold = [(k_1,f_1,t_1,T_1), \cdots ]$ is an array of quadruples with
the decreasing argument number $k_j$, the variable $f_j$, the term $t_j$ and
its type $T_j$. The terms $t_j$ might contain any variable $f_l$.


\paragraph{Positivity}

\begin{definition}
  The set $\strictpos_I$ of \emph{strict positive terms} over the variable $I$
  is defined as the least set satisfying the rules
  \begin{enumerate}

  \item
    $\ruleh{I \notin \freevars(T)}{T \in \strictpos_I}$

  \item
    $\ruleh{I \notin \freevars(\tbold)} {I \tbold \in \strictpos_I}$

  \item
    $\ruleh{I \notin \freevars(U),\: V \in \strictpos_I} {\forall(x:U) V \in
      \strictpos_I}$
  \end{enumerate}
\end{definition}

\begin{definition}
  The set $P_I$ of \emph{positive terms} over the variable $I$ is
  defined as the least set satisfying the rules
  \begin{enumerate}
  \item
    $\ruleh{I \notin \freevars(\tbold)} {I \tbold \in P_I}$

  \item
    $\ruleh{U \in \strictpos_I,\: V \in P_I}  {\forall(x:U) V \in P_I}$
  \end{enumerate}
\end{definition}






 \section{Inductive Definitions}
\label{sec:inductivedefintions}

We write an inductive definition as
$$\ind(\Gamma_P, \Gamma_I, \Gamma_C)$$
where
$$
\begin{array}{llll}
  \Gamma_P &=& [p_1:P_1, \cdots , p_{n_p}:P_{n_p}]
  &\text{Parameters}
  \\
  \Gamma_I &=& [I_1:T_{I1}, \cdots, I_{n_I}: T_{In_I}]
  &\text{Inductive types}
  \\
  \Gamma_C &=& [\Gamma_{C1}, \cdots, \Gamma_{Cn_I}]
  &\text{Constructors for each type}
  \\
  \Gamma_{Ci} &=& [c_{i1}:C_{i1}, \cdots, c_{in_{Ci}}: C_{in_{Ci}}]
  &\text{Constructors for }I_i
  \\
  C_{ij} &=& \forall(\tbold:\Tbold) I \pbold \abold
  & \text{Type of a constructor}
  \\
  n_P &:& \text{Number of parameters} \\
  n_I &:& \text{Number of inductive types} \\
  n_{Ci} &:& \text{Number of constructors for type } i
\end{array}
$$


A valid inductive definition in the valid context $\Gamma$ must satisfy
\begin{enumerate}
\item
  The context $\Gamma, \Gamma_P$ is valid i.e. each parameter type $P_i$ is a
  valid type.

\item $\Gamma, \Gamma_P \vdash T_{Ii}: s_i$ has to be valid for some sort
  $s_i$ and the normal form of each $T_{Ii}$ must be an arity of the sort
  $s_i$ i.e. has the form $\forall(\abold:\Abold) s_i$. $s_i$ is called the
  sort of the $i$-th inductive type of the definition.

\item
  $\Gamma, \Gamma_P, \Gamma_I \vdash C_{ij}: s_{ij}$ is valid for some sort
  $s_{ij}$, i.e. all constructor types $C_{ij}$ are valid types in a context
  where all inductive types are assumed.

\item Positivity: By definition the constructor types have the form
  $C_{ij} = \forall(\tbold:\Tbold) I_i \pbold\abold$. Conditions:
  \begin{enumerate}
  \item The arguments $\abold$ must not contain any $I_l$.

  \item All argument types $T_k$ of the array $\Tbold$ must satisfy one of
    the conditions:
    \begin{enumerate}

    \item $T_k$ does not contain any inductive type $I_l$.

    \item $T_k \to^* \forall(\ubold:\Ubold) I_m \pbold \abold$ for some $I_m$
      and no $I_l$ occurs in $\abold$ nor in $\Ubold$.

    \item $T_k \to^* J \pbold_J \abold_J$, $J$ is another inductive type which
      is not defined mutually, no $I_l$ occurs in $\abold_J$ (but might occur
      in $\pbold_J$) and all constructor types of $J$ instantiated with
      $\pbold_J$ must satisfy the same positivity conditions.
    \end{enumerate}
  \end{enumerate}

\item Predicativity: If the sort of an inductive type is $\Set$, then no
  argument type of the constructors can be $\Set$ nor $\Type$. If this were
  allowed, we could feed the inductive type as an argument into its own
  constructor which would make the sort $\Set$ impredicative.
\end{enumerate}









\section{Scratch}




\begin{lstlisting}
  Inductive tree (A:Type): Type :=
  | Node: A -> list (tree A) -> tree A.

  Inductive list (A:Type): Type :=
  | Nil: list A
  | Cons: A -> list A -> list A.
\end{lstlisting}

\end{document}
